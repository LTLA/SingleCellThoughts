\documentclass{article}
\usepackage[margin=3cm]{geometry}
\usepackage{natbib}
\usepackage{amsmath}
\usepackage{bm}

% document begins here
\begin{document}
\vspace*{0.35in}

% title goes here:
\begin{flushleft}
{\Large
\textbf\newline{Detecting clusters of doublet cells in \textit{scran}}
}
\newline

% authors go here:
%\\
Aaron T. L. Lun\textsuperscript{1,*}
\\
\bigskip
\bf{1} Cancer Research UK Cambridge Institute, University of Cambridge, Li Ka Shing Centre, Robinson Way, Cambridge CB2 0RE, United Kingdom
\\
\bigskip
* aaron.lun@cruk.cam.ac.uk

\end{flushleft}

\section{Mathematical framework for doublets}
Consider a cell population $i$ that has mean transcript count $\lambda_{gi}$ for gene $g$.
Assume that each population exhibits a unique scaling bias $s_i$.
The observed read/UMI count for each gene is then $\mu_{gi}=s_i\lambda_{gi}$.
(For simplicity, we will ignore gene-specific scaling biases, as this is easily accommodated by considering $\lambda_{gi} \equiv \phi_g \lambda_{gi}$ for some bias $\phi_g$.)
The expected total count for each population is $N_i = \sum_g \mu_{gi}$.

Now, let us consider a doublet population $j$ that forms from two parent populations $i_1$ and $i_2$.
The observed read count for $g$ in $j$ is $\mu_{gj} = s_j (\lambda_{gi_1} + \lambda_{gi_2})$.
Note that $s_j$ need not be any particular function of $s_{i_1}$ and $s_{i_2}$.
This relationship depends on how quickly the reverse transcription and amplification reagents are saturated during library preparation, which is difficult to make assumptions around.

\section{Normalization by library size}
We obtain log-normalized expression values for each cell based on the library size, rather than other size factors.
This is because we are interested in the (proportional) composition of each cell.
Specifically, the composition of a doublet cluster should be intermediate between the two source clusters.

Assume that the library size-normalized expression values are such that $\mu_{gi_1}N_{i_1}^{-1} < \mu_{gi_2}N_{i_2}^{-1}$, 
i.e., the proportion of $g$ increases in $i_2$ compared to $i_1$.
The contribution of each $s_i$ cancels out, yielding
\[
\frac{\lambda_{gi_1}}{\sum_g \lambda_{gi_1}} < \frac{\lambda_{gi_2}}{\sum_g \lambda_{gi_2}} \;.
\]
The normalized expression value of the doublet cluster $j$ is subsequently 
\[
\frac{\lambda_{gi_1} + \lambda_{gi_2}}{\sum_g (\lambda_{gi_1} + \lambda_{gi_2})} \;,
\]
and it is fairly easy to show that, under the above assumption, we have
\[
\frac{\lambda_{gi_1}}{\sum_g \lambda_{gi_1}} < 
\frac{\lambda_{gi_1} + \lambda_{gi_2}}{\sum_g (\lambda_{gi_1} + \lambda_{gi_2})}  < 
\frac{\lambda_{gi_2}}{\sum_g \lambda_{gi_2}} \;.
\]
In other words, the normalized expression of the doublet cluster lies between the two parents.

It is harder to guarantee the intermediacy condition with arbitrary size factors, which is why we only use the library sizes for normalization instead.
The exception is that of spike-in size factors that would estimate $s_i$ directly. 
This would allow us to obtain estimates of $\lambda_{gi}$ for the parent clusters and of $\lambda_{gi_1} + \lambda_{gi_2}$ for the doublets.
In this manner, we could more precisely identify doublet clusters as those where the normalized expression value is equal to the sum of the parents.
Unfortunately, spike-ins are generally not available for droplet-based data sets where doublets are most problematic.

\section{Testing for (lack of) intermediacy}
We want to identify the clusters that may be comprised of doublets of other clusters.
For each cluster $j'$, we test for differential expression in the library size-normalized expression profiles against every other cluster $i'$.
For each pair of other clusters $i'_1$ and $i'_2$, we identify genes that change in $j'$ against both $i'_1$ and $i'_2$ \textit{in the same direction}.
The presence of such genes violates the intermediacy expected of a doublet cluster and provides evidence that $j'$ is not a doublet of $i'_1$ and $i'_2$.

Significant genes are identified by an intersection-union test on the $p$-values from the pairwise comparisons between $j'$ and $i'_1$ or $i'_2$. 
(Specifically, $t$-tests are used via the \texttt{findMarkers} function.)
The $p$-value for a gene is set to unity when the signs of the log-fold changes are not the same between comparisons.
Multiple correction testing is applied using the Benjamini-Hochberg method, and the number of genes detected at a specified false discovery rate (usually 5\%) is counted.
The pair $(i'_1, i'_2)$ with the fewest detected genes are considered as the putative parents of $j'$.

In theory, it is possible to compute the Simes' combined $p$-value across all genes to reject the doublet hypothesis for $j'$.
This would provide a more rigorous approach to ruling out potential doublet/parent combinations.
However, this is very sensitive to misspecification of clusters -- see below.

\section{Calling doublet clusters}
Assuming that most clusters are not comprised of doublets, we identify clusters that have an unusually low number of detected genes that violate the intermediacy condition.
This is achieved by identifying small outliers on the log-transformed number of detected genes, using the median absolute deviation-based method in the \texttt{isOutlier} function.
(We use a log-transformation simply to improve resolution at low values.)
Clusters are likely to be doublets if they are outliers on this metric. 

Doublet clusters should also have larger library sizes than the proposed parent clusters.
This is consistent with the presence of more RNA in each doublet, though the library size of the doublet cluster need not be a sum of that of the parent clusters
(due to factors such as saturation and composition effects).
The proportion of cells assigned to the doublet cluster should also be ``reasonable'';
exactly what this means depends on the experimental setup and the doublet rate of the protocol in use.

\section{Discussion}
We have chosen the above approach due to its lack of assumptions.
Specifically, we do not need to know $s_i$ for each cluster in order to satisfy the intermediacy condition.
This allows us to identify doublets even when the exact location of the doublet is unknown.
In contrast, methods that attempt to reconstruct the doublet cluster need to know the ratio in which RNA from the parent populations are mixed.
This requires knowledge of the total RNA content of each cell, which is not available without spike-ins;
and $N_i$ is a poor substitute, especially in experiments that reach saturation.

The intermediacy condition also provides robustness against misspecification of the clusters.
Consider the situation where an adjacent population $x$ is merged into the doublet or parent clusters.
The intermediacy condition would still hold for genes where the log-fold change between $x$ and its merged partner population is smaller than the log-fold change between the parents.
The condition would only fail in extreme cases, e.g., for strong marker genes in $x$ that now appear to be uniquely expressed in the doublet cluster,
or due to apparent downregulation of genes in the parent cluster when the cluster average includes a non-expressing population.
The number of such genes should be low if $x$ was similar enough to be merged in the first place.
This means that the doublet cluster should still be detected as a small outlier for the number of detected genes.

\end{document}
